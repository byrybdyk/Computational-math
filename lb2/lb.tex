\documentclass[14pt]{article}

\usepackage[utf8]{inputenc}
\usepackage[russian]{babel}
\usepackage{geometry}
\usepackage{graphicx}
\usepackage{extsizes}
\usepackage{multicol}
\setlength{\columnsep}{1cm}
\usepackage[unicode, pdftex]{hyperref}
\usepackage{listings}
\usepackage{enumerate}
\usepackage{mdframed}

\geometry{a4paper, top=20mm, bottom=20mm, left=20mm, right=10mm}

\lstdefinestyle{output_style}{
    basicstyle=\ttfamily
}

\lstset{style=output_style}

\begin{document}

\thispagestyle{empty}

\begin{center}
    Федеральное государственное\\
    автономное учебное учреждение высшего образования\\
    <<Национальный исследовательский университет ИТМО>>\\
\vspace{0.5cm}
    Мегафакультет компьютеных технологий и управления\\
    Факультет программной инженерии и компьютерной техники
\end{center}

\vspace{3cm}

\begin{center}
\Large
\textbf{
    Отчёт\\
    по лабораторной работе №2\\
    по дисциплине <<Вычислительная математика>>
}
\end{center}

\begin{center}
\large
    Вариант 5
\end{center}

\vspace{5cm}

\begin{flushright}
    Группа: P3218\\
    \vspace{0.2cm}
    Студент: Зарубов Егор Николаевич\\
    \vspace{0.2cm}
    Преподаватель: Бострикова Дарья Константиновна
\end{flushright}

\vspace{5cm}

\begin{center}
    Санкт-Петербург\\
    2024
\end{center}

\newpage

\tableofcontents

\setcounter{page}{0}
\thispagestyle{empty}

\newpage

\section{Цель работы}

Целью работы является знакомство с численными методами решения СЛАУ, а также способами и особенностями их реализации на языках программирования.

\section{Задание}

Написать программу для решения СЛАУ с использованием метода Гаусса. Требования к программе:
\begin{itemize}
    \item В программе численный метод должен быть реализован в виде отдельной подпрограммы/метода/класса, в который исходные/выходные данные передаются в качестве параметров.
    \item Размерность матрицы $n\leq20$ (задается из файла или с клавиатуры - по выбору конечного пользователя).
    \item Должна быть реализована возможность ввода коэффициентов матрицы, как с клавиатуры, так и из файла (по выбору конечного пользователя).
    \item Должно быть реализовано: вычисление определителя, вывод треугольной матрицы (включая преобразованный столбец В), вывод вектор неизвестных ($x_1$, $x_2$, ..., $x_n$), вывод вектора невязок ($r_1$, $r_2$, ..., $r_n$).
\end{itemize}

\section{Описание метода выполнения}
Программа, реализующая метод Гаусса для решения СЛАУ, была написана на языке программирования Python версии 3.11. Были использованы библиотеки \textit{random} (для генерации случайной матрицы), \textit{os} (для проверки файлов),  \textit{fractions} и \textit{decimal} (для проведения точных вычислений).

\newpage

\section{Исходный код программы}

Репозиторий на GitHub: \href{https://github.com/byrybdyk/Computational-math/tree/main/lb1}{\underline{ссылка}}

\begin{mdframed}
\footnotesize
\begin{verbatim}

def method_Gauss(matrix):
    for i in range(len(matrix)):
        if matrix[i][i] != 0:
            for j in range(i + 1, len(matrix)):
                buffer_row = string_multiplication(matrix, i, matrix[j][i], matrix[i][i])
                matrix = string_addition(matrix, j, buffer_row)
        else:
            for p in range(i + 1, len(matrix)):
                if matrix[p][i] != 0:
                    swap_rows(matrix, i, p)
                    break
                else:
                    raise ValueError("No nonzero diagonal element")
            for j in range(i + 1, len(matrix)):
                buffer_row = string_multiplication(matrix, i, matrix[j][i], matrix[i][i])
                matrix = string_addition(matrix, j, buffer_row)

    return matrix

\end{verbatim}
\end{mdframed}
\begin{center}
    Листинг 1: Метод получения треугольной матрицы
\end{center}

\begin{mdframed}
\small
\begin{verbatim}

def reversal_method(matrix):
    n = len(matrix)
    x = [0] * n  # Create a list to store solutions

    for i in range(n - 1, -1, -1):  # Start from the last equation
        sum = 0
        for j in range(i + 1, n):  # Iterate over coefficients above the diagonal
            sum += matrix[i][j] * x[j]
        x[i] = (matrix[i][-1] - sum) / matrix[i][i]  # Compute the variable value
    x = list(map(float, x))
    return x

\end{verbatim}
\end{mdframed}
\begin{center}
    Листинг 2: Метод получения вектора решений
\end{center}

\newpage

\section{Рассчётные формулы}

Метод Гаусса состоит из следующих этапов:

\begin{enumerate}[I.]
    \item Прямой ход -- приведение матрицы к треугольному виду.

    Для каждого $i$-го уравнения (от $1$ до $n-1$) исключаем из каждого последующего за ним $j$-го уравнения $i$-ую компоненту искомого вектора $X$ при помощи метода единственного деления:
    $$
    M_j = M_j - \frac{a_{ji}}{a_{ii}} \cdot M_i
    $$
    Если оказывается, что в текущем $i$-ом уравнении ведущий коэффициент $a_{ii}$ равен 0, то переставляем текущее уравнение местами с первым уравнением, где этот коэффициент не нулевой. Если такого уравнения нет, то единтсвенное решение найти не удастся.

    \item Вычисление определителя треугольной матрицы.
    
    После приведения матрицы к треугольному виду определить вычисляется произведением ведущих элементов. 
    $$
    D = (-1)^p \prod \limits_{i = 1}^{n} a_{ii}
    $$
    Возможные перстановки $p$ на первом этапе могут повлечь за собой смену знака определителя. 

    Если вычисленный определитель равен 0, то единтсвенное решение найти не удастся.
    \item Обратный ход -- вычисление компонент искомого вектора $X$.
    $$
    x_i = \frac{1}{a_{ii}} (b_i - \sum \limits_{k=i+1}^{n} a_{ik}x_k)
    $$
\end{enumerate}

Чтобы оценить правильность полученного решения, его подставляют в систему и сравнивают результат с исходынм вектором свободных членов. Вычисляют вектор невязки $R$:
$$
R = AX - B
$$

\newpage

\section{Пример вывода программы}

\begin{mdframed}
\small
\begin{verbatim}

Choose the command:
1) Manual matrix input     
2) Input matrix from file  
3) Generate a random matrix
4) Exit the program        
3
Enter the dimension of the matrix, not exceeding 20
4
Matrix read:
96 88 27 23 | 79
82 90 87 73 | 48
77 92 18 59 | 33
99 25 66 80 | 23

rows swap =  0

Matrix after forward pass:
96.0 88.0 27.0 23.0 | 79.0
0.0 14.833333333333334 63.9375 53.354166666666664 |
-19.479166666666668
0.0 0.0 -95.97050561797752 -36.48174157303371 | -2.240168539325843
0.0 0.0 0.0 170.5401501514686 | -152.3178425604777

Determinant of the matrix =  -23306358

x1 = 0.6274798919676768  x2 = 0.3353086312327306  x3 = 0.36285956819164966
x4 = -0.8931494573283393
r1 = 0.0  r2 = -3.552713678800501e-15  r3 = -6.661338147750939e-15  r4 = 0.0

\end{verbatim}
\end{mdframed}
\begin{center}
    Листинг 3: Матрица сгенерирована случайно
\end{center}

\newpage

\begin{mdframed}
\begin{verbatim}
Choose the command:
1) Manual matrix input     
2) Input matrix from file  
3) Generate a random matrix
4) Exit the program        
2
Enter the relative path to your file
1.txt

Matrix read:   
10 -7 0 | 7    
-3 2 6 | 4     
5 -1 5 | 6     

rows swap =  0 

Matrix after forward pass:
10.0 -7.0 0.0 | 7.0
0.0 -0.1 6.0 | 6.1
0.0 0.0 155.0 | 155.0

Determinant of the matrix =  -155

x1 = 0.0  x2 = -1.0  x3 = 1.0
r1 = 0.0  r2 = 0.0  r3 = 0.0

\end{verbatim}
\end{mdframed}
\begin{center}
    Листинг 4: Матрица считана из файла
\end{center}

\newpage

\section{Вывод}

В ходе выполнения текущего проекта были изучены численные методы решения систем линейных алгебраических уравнений (СЛАУ), как прямые, так и итерационные. На основе метода Гаусса была разработана программа на языке Python для решения СЛАУ. Были проанализированы условия применимости различных методов, учитывая, например, что метод Гаусса неприменим, если определитель матрицы равен нулю.

Выявлены достоинства и недостатки метода Гаусса по сравнению с итерационными методами: метод Гаусса прост и универсален, обеспечивает точные решения за конечное число операций, но требует больше памяти и накапливает погрешности. Итерационные методы более оптимальны с точки зрения использования памяти и не накапливают погрешности, но их реализация сложнее и они не всегда дают полностью точные ответы.

Была проведена работа по выбору оптимального метода для точных вычислений и визуализации числовых данных. Также значительное внимание уделено проектированию и разработке удобного консольного интерфейса программы, обеспечивающего удобство в использовании и обработку различных ситуаций, включая критические.

\end{document}
